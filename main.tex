\documentclass[12pt]{curriculum-vitae} 
\usepackage[a4paper, total={7.3in, 9.7in}]{geometry} 
\usepackage{hyperref} 

\hypersetup{
    colorlinks=true,
}

\name{Abderahmane Bouziane}
\mailingaddr{~}
\desc{pursuit of clarity}
\webpage{\href{https://academic-website-lovat.vercel.app/}{links}}
\phone{~}

\begin{document}

\maketitle

\cvsection[Education]{
    \begin{detail}[Université de Montréal]{Sep 2022}{May 2025}
        B.Sc in Mathematics and Physics\\
        % 3.83 / 4.3\\
        Mention d'excellence
    \end{detail}
    \begin{detail}[Polytechnique Montréal]{Sep 2016}{Dec 2020}
        B.Eng Software Engineering\\
    \end{detail}
}


\cvsection[Research]{
    \begin{detail}[Meta-hahn algebra to polynomials]{}{Summer 2023}
        Operator algebra\\
        Generalized eigen value problem\\
        Bi-orthogonal polynomials \href{https://arxiv.org/pdf/2009.05905.pdf}{Meta-Hahn}\\
        Askey scheme\\
        Mathematica\\
        Under the supervision of Luc Vinet
    \end{detail}
}

\cvsection[Professional expereience]{
    \begin{detail}[Squarepoint Capital | Hedge fund]{May 2020}{Mar 2023}
        Systems programming.\\
        Market Data.\\
        c++, python, linux.
    \end{detail}
}

\cvsection[Teaching Assistant]{
    \begin{detail}[Université de Montréal ]{}{Fall 2023}
        \href{https://admission.umontreal.ca/cours-et-horaires/cours/phy-1902/}{PHY-1902} - Electricity and magnetism
    \end{detail}
    \begin{detail}[Polytechnique Montréal ]{}{Fall 2019}
        \href{https://www.polymtl.ca/programmes/cours/structures-de-donnees-et-algorithmes}{INF-2010} - Data structures and algorithms
    \end{detail}
}

\cvsection[Scolarships]{
    \begin{description}
        \item \href{http://ism.uqam.ca/~ism/accueil/bourses\#4}{ISM}
              \\
              Summer undergrad research  2023 \\
    \end{description}
}

\cvsection[Personal projects]{
    \begin{description}
        \item
              \href{https://github.com/honeyspoon/music_generation}{Music generation}\\
              Using randomness constrained by rules of music theory to generate melodies.\\
              Outputs a midi stream that can be interpreted by any DAW (Logic) to play sounds.

        \item
              \href{https://github.com/honeyspoon/cal-udem}{Calendar generation}\\
              Scrape university website to get the class schedules\\
              Displays them in a calendar view and allows for exporting to a google calendar friendly format
    \end{description}
}
\end{document}